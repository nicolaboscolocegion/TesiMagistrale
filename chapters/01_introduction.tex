%!TEX root = ../main.tex

\chapter{Introduction}
\label{chp:intro}

\section{Case introduction}

At the university hospital of Padua, 
there is a very important cardiac surgery center where various heart interventions are performed on many people,
the need to teach and visualize the case of operation is very important. \\
Thanks to MRI, doctors can see not only pictures of the heart, but also can even make 3D models.
Each 3D model can show every detail of the patient heart, this can make surgeons able to analyze and show where and how to resolve the problem.

\subsection{Virtual reality head mounted displays}

\paragraph{The equipment:}
The university of Padua has some Meta quest 2\footnote{All rights to meta reserved} [fig:\ref{fig:metaQuest2}] a \ac{HMD} for \ac{VR}. \\
The Meta Quest 2 is a standalone \ac{HMD}, this means that they don't need other peripherals like an external console or \ac{PC} for working.\\
For navigating, the Meta Quest 2 uses two wireless controller, but it also has the possibility to just use your own hands to navigate the interfaces.
For the context awareness it uses 4 infrared cameras and sensors like multi axes gyroscopes and accelerometers.\\
They use a custom version of the Android \ac{OS}, this can give a certain degree of liberty in creating APPs for the device.

\begin{figure}[h]
  \centering
  \includegraphics[width=0.5\textwidth]{metaQuest2.jpg}
  \caption{Meta quest 2}
  \label{fig:metaQuest2}
\end{figure}

\paragraph{Use cases of VR:}
Principally the surgeons are using VR equipment for training and showing critical health conditions of different patients hearts.
They are using an APP called Shapes XR, this APP has a web interface for upload 3D models and then show them on the \ac{HMD}.
The app has a multiplayer functionality so that multiple people can look at the 3D models in the environment, even if the developers recommend at max 8 people, they tested with 14 users connected and there weren't any problems.

\subsection{APP problems}

\paragraph{How Shapes XR works:}
Shapes XR lets you create rooms, accessible via a code, were multiple people can create 3D models with basic tools like 3D brushes and standard shapes like cubes, pyramids and so on.
It also let you upload a 3D model file on their own website, so that in the home you can download it and start to work on it.
It let you also create your own avatar.
This is the main feature that the surgeons are using for show the 3D Models

\paragraph{The Problems:}
Unfortunately Shapes XR is principally used for 3D modelling, so the app has a lot of features like changing the scale of the world or brushes for modelling the objects that aren't useful for the surgeons,
and quite distracting because for a lot of people is the first time using a \ac{HMD}.\\
The user experience is extremely important in \ac{VR} because is difficult to tutoring the user while is using the \ac{HMD}.
Then Shapes XR position the user in an empty 3D plane with little to none point of reference so If a user accidentally uses the teleport function, they may find themselves somewhere far away from the scene they are supposed to be watching.

\paragraph{Feedbacks from surgeons and nurses:}

The 12/12/2023 I have whiteness a lesson with the integration of the Quest 2 and Shapes XR.
First it was pretty chaotic, a lot of people didn’t even know how to use the controllers, and they had problems even for putting the code for entering the session.
Unfortunately we didn’t have time to make a nice lesson for teaching how to use the \ac{HMD}, the tutorial made by Meta approximates takes 15min to complete, even more if the user want to try the mini-games, so we did not have the time to show it.
The main critical points were:

\begin{itemize}
  \item Inadequate tutoring for teaching how to use the \ac{HMD}
  \item Difficulty for accessing the multiplayer room
  \item Difficulty at moving in the room
  \item Some people avatar were blocking the visual of some people
\end{itemize}
\noindent
How we can see a new app for this use case could be useful, and this is the main reason why this project exist.