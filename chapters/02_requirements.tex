%!TEX root = ../main.tex

\chapter{Requirements}
\label{chp:Requirements}
\noindent
This chapter outlines the software features and constraints, specifying what the system must accomplish and the characteristics it must meet to fulfill users needs.

\section{Functional Requirements}
\noindent
The main features of the software must be:

\begin{itemize}
  \item \textbf{Show custom 3D model at runtime:} The software must be able to download 3D models in OBJ format and render them at runtime, which will also show coloring made by a surgeon. The 3D model should also be movable and have the possibility to change its size. 
  \item \textbf{Multiplayer functionality:} The software must recreate a 3D virtual classroom where a professor can show the 3D model to the students, all students and professor must be synchronized.
  \item \textbf{Upload of 3D models:} There will be a wen app where a surgeon can upload and preview 3D models.
  \item \textbf{Tutoring functionality:} The software will have some tools for learning such as laser pointers, and the possibility to change the position and dimension of the 3D model.
  \item \textbf{Compatibility with Meta quest 2:} The system will need to run on Meta quest 2,as these are the HMD available at present in \ac{AOUPD}.
\end{itemize}
\section{Data requirements}
\noindent
The system necessitates the following data (with the following characteristics) to fulfill its objectives:

\begin{itemize}
  \item \textbf{Required Data:} 
  \begin{itemize}
    \item 3D models in OBJ format.
    \item Local \ac{IP} address for multiplayer.
    \item Generating and managing codes for multiplayer sessions.
  \end{itemize}
  \item \textbf{Data Format:} All data must respect its standard, other communication between clients and server must use \ac{JSON} format file. These formats facilitate data exchange with external systems.
\end{itemize}

\section{Non-functional requirements}
\noindent
To ensure the development of an effective \ac{VR} software, the following non-functional requirements must be considered:

\begin{itemize}
  \item \textbf{Use of Game engine:} A game engine is a software made by another company that is capable of creating a video game, by giving the developers some tools for making the development experience quicker and easier.
  \item \textbf{Simple Back end:} The backend must do the least things effectively because the IT department does not need to update the server that will host the services.
  \item \textbf{Compliance with Internet standards:} The system will be compliant with \ac{HTTP} for all communication between server and client, the \ac{HMD} will use the game engine multiplayer system.
  \item \textbf{Maintaining performance:} \ac{VR} applications need to perform extremely well for having a good \ac{VR} experience, Meta allows a minimum of 72 \ac{FPS}, but our target for a better experience will be 90 \ac{FPS}.
\end{itemize}

