%!TEX root = ../main.tex

\chapter{The project}
\label{chp:project}
\noindent
This chapter will explain the development behind the project by starting from the fundamentals.
\section{The 3D models}
\noindent
For understanding how \ac{UE} will show 3D models and how they will be stored, we must talk about their characteristics.

\begin{itemize}
    \item \textbf{Vertices:} points that describe the geometry
    \item \textbf{Faces} indicated were there is a polygon by grouping 3 or more vertices 
    \item \textbf{Normal:} there is one for each vertex that is in a face, indicates the direction to which the face is exposed, and for calculating how light is reflected
    \item \textbf{UV:} vectors that helps how a texture should be applied to the model
    \item \textbf{Vertex colors:} RGB vector that indicates a color for each vertex
\end{itemize}
\noindent
As we talked about in chapter \ref{chp:Requirements}, we will use the OBJ file format for storing files.
The React Three Fiber has a native support for OBJ, and the backend server does not need to read the file but just to manage by saving, deleting and send it via \ac{HTTP}.
Unfortunately \ac{UE} does not have a OBJ file reader usable at runtime but just an importer for what it calls static meshes (3D models that don't have moving parts).
So there is the need to build a parser OBJ to \ac{UE} custom types.\\
First we need to understand how the OBJ file format is compose of, here a general example code:\ref{code:OBJExample}

\lstinputlisting[language=Octave, caption=OBJ file example, label={code:OBJExample}]{code/exampleOBJ.txt}
\noindent
The vertices, UV and normals are simply written, instead the faces have different formats, first not always they use triangles, but also quads, this depends on how the file was exported.
For ease of use the parser will support both. The numbers of the face are the indices of vertex, indices starts from 1.
Then a face can have also the UV and normals corresponding for the vertex. For our use cases UV aren't needed, but for future-proof they are still been parsed correctly.\\
Sometime is useful to divide the 3D model in multiple objects the OBJ format represent by dividing the 3D model with a "\verb|o|".
OBJ can also divide the faces in groups by dividing them with a "\verb|g|".
Here an example of how the division works: code:\ref{code:OBJgrouping}

\lstinputlisting[language=Octave, caption=OBJ grouping, label={code:OBJgrouping}]{code/exampleOBJGrups.txt}
