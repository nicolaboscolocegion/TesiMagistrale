%!TEX root = ../main.tex

\chapter{Future work}
\label{chp:conclusions}
\noindent
In this thesis, we successfully designed, developed, and tested a comprehensive software solution that meets all the requirements specified by the surgeons.\\
As with any software development software, there is always room for additional features and refinements. The following paragraphs outline potential future work to enhance the functionality and overall outcome of this software.

\section{Publication in the store}
\noindent
The software is still in its prototype stage, so it requires further attention before it can be published on the Meta Quest Store.
The server is currently in its prototype stage and needs to implement user authentication. It must also be deployed in a public domain to allow users outside the university to access the software.
Additionally, it should be scalable, so it can manage a number of users that the server is currently unable to support.


\section{Future features}
\noindent
The software could benefit from additional features that are described in this section.

\paragraph{Slicing 3D models}
so that surgeons could gain deeper insights from the 3D model if it could be sliced along planes, similar to \ac{CT} scans and \ac{MRI}.

\paragraph{Real time coloring}
could be done during a lecture, to highlight specific surfaces of the heart by coloring them to emphasize particular areas.

\paragraph{Visualization of other media}
by leveraging the capabilities of \ac{VR} to display various types of media, it will be beneficial to include images or videos of \ac{CT} scans and \ac{MRI}, providing surgeons with multiple resources to work with.

\paragraph{Procedural \ac{LOD} for 3D models}
because the 3D models used are quite demanding in terms of the number of vertices and triangles. If the backend could dynamically generate \ac{LOD}s, the \ac{HMD} would benefit from significant performance gains.

\paragraph{3D models caching}
because lectures can be highly dynamic, requiring the display of various heart models. Additionally, previously used 3D models could be downloaded and cached on the \ac{HMD}, reducing the time needed to load and display them.

\paragraph{Support for new devices}
At the time of writing, Meta has also released two new devices, the Meta Quest 3 and 3s. Although they use the same \ac{OS}, it will be necessary to conduct testing to verify compatibility with these devices.


\chapter{Conclusions}
\noindent
The development of the \ac{VR} app enabled us to create an immersive training experience for cardiac surgeons.
The process was challenging due to long development cycles, primarily caused by the need to compile \ac{APK} files for testing on the \ac{HMD}.\\
The software not only represents an application of the \ac{VR} technology, but also the features  that the \ac{UE} has in developing not only \ac{VR} applications but also in general 3D applications.\\
As explained in Chp.[\ref{chp:conclusions}] the software is far from finished but it is an opportunity for the university of Padua to create a unique application for teaching.\\