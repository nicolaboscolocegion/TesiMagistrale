%!TEX root = ../main.tex

\chapter{Future work}
\label{chp:conclusions}
\noindent
In this thesis, we successfully designed, developed, and tested a comprehensive software solution that meets all the requirements specified by the surgeons.\\
As with any software development software, there is always room for additional features and refinements. The following paragraphs outline potential future work to enhance the functionality and overall outcome of this software.

\subsection{Publication in the store}
\noindent
The software is still in its prototype stage, so it requires further attention before it can be published on the Meta Quest Store.
The server is currently in its prototype stage and needs to implement user authentication. It must also be deployed in a public domain to allow users outside the university to access the software.
Additionally, it should be scalable, so it can manage a number of users that the server is currently unable to support.
To make this software more desirable to a larger audience, it could be enhanced with additional features, as explained in the following sections.



\subsection{Slicing 3D models}
\noindent
At present the app allows viewing the model from the exterior.
To view it from the interior, the user must "walk" inside the model, to improve this experience a special cut out model should be created and loaded so that different parts of the heart can be seen individually.
Another useful way to view the inside of models could be a tool operated by the teacher or user that allows to slice the model along planes so that surgeons could gain deeper insights from the 3D model if it could be sliced along planes, like \ac{CT} scans and \ac{MRI}, this could be done locally thanks to \ac{UE} functions.

\subsection{Real time coloring}
\noindent
At present, the app supports colored models, but the colors need to be set before the rendering and, as described before, they are encoded in the model itself.
As heart models can be quite complex, a real time coloring tool could be a useful tool during a lecture, to highlight some surfaces of the heart by coloring them to emphasize particular areas.

\subsection{Visualization of other media}
\noindent
By leveraging the capabilities of \ac{VR} to display various types of media, it will be beneficial to include images or videos of \ac{CT} scans and \ac{MRI}, providing surgeons with multiple resources to work with.
Thanks to \ac{VR} this media can be shown in very different ways, we could build a simple virtual monitor to see them but also be more creative by recreating in \ac{VR} devices like tablets or smartphones.

\subsection{Procedural \ac{LOD} for 3D models}
\noindent
Because the 3D models used are quite demanding in terms of the number of vertices and triangles. If the backend could dynamically generate \ac{LOD}s, the \ac{HMD} would benefit from significant performance gains.
The algorithms required for creating \ac{LOD}s are quite demanding in time execution, this is why it should be done on the server.
Additionally, the OBJ file format is not suitable for storing such data, so transitioning to a more advanced file format like FBX or GLB would be advisable.

\subsection{3D models caching}
\noindent
Because lectures can be highly dynamic, requiring the display of various heart models. Additionally, previously used 3D models could be downloaded and cached on the \ac{HMD}, reducing the time needed to load and display them.
To implement it, we need to create some caching rules and update the server \ac{API}s for enabling this functionality.

\chapter{Conclusions}
\noindent
The development of the \ac{VR} app enabled us to create an immersive training experience for cardiac surgeons.
The process was challenging due to long development cycles, primarily caused by the need to compile \ac{APK} files for testing on the \ac{HMD}.\\
The software not only represents an application of the \ac{VR} technology, but also the features  that the \ac{UE} has in developing not only \ac{VR} applications but also in general 3D applications.\\
As explained in Chp.[\ref{chp:conclusions}] the software is far from finished but it is an opportunity for the University of Padua to create a unique application for teaching.\\