%!TEX root = ../main.tex

\chapter{Performance and optimization}
\label{chp:performance}
\noindent
Running a \ac{VR} application is highly resource-intensive. Not only must the processor handle all sensor data, but it also has to run a 3D application that is rendered on two screens, aiming to achieve at least a stable 72 \ac{FPS} experience.
On top of that, the Meta Quest 2 has limited mobile hardware Tab.[\ref{tab:specs}]. This chapter will explain all the optimizations necessary to run the application at 90 \ac{FPS}, as well as demonstrate the performance of the OBJ parser build.


\begin{table}
  \centering
  \begin{tblr}{
      colspec = {|l|l|},
      rows={valign=m},
      row{1}={font=\bfseries}
    }
  \hline
  Charateristic & Specs                                                          \\ \hline
  Chipset       & Qualcomm Snapdragon XR2                                       \\ \hline
  CPU           & Octa-core Kryo 585 (1$\times$2.84 GHz, 3$\times$2.42 GHz, 4$\times$1.8 GHz)  \\ \hline
  GPU           & Adreno 650                                                    \\ \hline
  RAM           & 6 GB LPDDR5                                                    \\ \hline

  \end{tblr}
  \caption{Meta Quest 2 specs}
  \label{tab:specs}
\end{table}

\section{Meta XR performance check}
\noindent
Thanks to the Meta XR plugin, Meta directly suggests certain options to enable or disable. The most important aspects will be explained in the next paragraphs.

\paragraph{Post-processing}
must be disabled because it can be very demanding on the device. As discussed in Chapter \ref{Chp:fade}, mobile HDR is already turned off, however, \ac{UE} still enables lens flares by default. 
To disable it, we need to place a \texttt{Post Process Volume} in the scene and set the lens flare intensity to \texttt{0}.

\paragraph{Dynamic lighting}
must be disabled. This option is probably the most impactful. When enabled, the app runs at less than 60 \ac{FPS}, but with it disabled, performance reaches a stable 90 \ac{FPS}, yielding an improvement of over 30 \ac{FPS} and providing a more stable experience overall.
Unfortunately, dynamic lights can enhance immersion, however, the only moving elements are the \texttt{VR Characters} and the \texttt{3D Model Viewer}, so the absence of dynamic lighting will not be very noticeable.\\
Static lighting will still be used, allowing the environment to have shadows cast by a directional light that simulates the sun, as well as by rectangular and point lights used for building interiors.

\paragraph{Half precision float for materials and shaders.}
As is well known, using smaller float values can improve system performance by sacrificing some precision, though fortunately, the side effects are often negligible.
However, in our case, this optimization has limited impact because the majority of materials used are static.

\subsection{Other optimizations for Meta quest}
\noindent
In addition to the previously described options, there are some additional settings that can help boost performance. These are simple Android manifest tags that enable the \ac{HMD} to activate additional functionality:

\begin{itemize}
  \item \textbf{Suggested CPU and GPU levels:} This setting allows the \ac{HMD} to operate at maximum performance levels.
  \item \textbf{Processor favor:} This option lets us choose whether the \ac{HMD} should prioritize maximum performance for either the GPU or CPU. This is useful because the device has limited power and heat dissipation, preventing it from running at 100\% performance continuously without the risk of throttling.
  \item \textbf{Enable dual core:} Normally, the \ac{HMD} uses only one core for the foreground app, but a meta tag enables a second core for background activities and parallel operations. This is particularly beneficial for the parser and asynchronous operations needed by \ac{UE}.
\end{itemize}

\section{Optimizations for Unreal engine VR}
\noindent
In addition to the options provided by the Meta XR performance check, there are settings within \ac{UE} (Ref.\cite{UEperformance}) that can further optimize the app for the \ac{HMD}.These optimizations are explained in the following paragraphs.

\paragraph{Instanced stereo}
When rendering a frame for an \ac{HMD}, the screens are typically managed as two separate entities. This is demanding for the GPU, as it has to render two frames simultaneously.
However, with Instanced Stereo, we can generate both views in a single pass. \ac{UE} achieves this by applying a small transformation to the loaded vertices, correcting the view difference between the left and right eyes. Additionally, this transformation is applied to shaders as well.
This rendering method relies on low-level \ac{API}s that enable it, the Meta Quest 2 uses the Vulkan graphics \ac{API}, which supports this feature.

