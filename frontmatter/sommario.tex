\begin{abstract}[it]
  Questa tesi descrive la progettazione e lo sviluppo di un'applicazione di realtà virtuale per il visore Meta Quest 2 per la formazione di medici ed infermieri nell'ambito della cardiochirurgia pediatrica.
  Lo scopo della applicazione è la presentazione di casi studio tramite la visualizzazione immersiva di modelli 3D derivati da tecniche di imaging.
  L'applicazione permette di creare classi virtuali dove gli studenti possono interagire con modelli 3D di cuori fisiologici o patologici con la supervisione del docente o in autonomia.
\end{abstract}